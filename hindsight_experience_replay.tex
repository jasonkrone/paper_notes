\section{hindsight Experience Replay by Marcin Andrychowicz et al. 2018}

\subsection{Summary:}

Hindsight Experience Replay (HER) is a technique related to reward shaping, which
"replays" a trajectory with a set of "goals" that may differ from the actual goal 
in order to improve learning efficiency. A design choice that must be made when using her
is what additional goals to use. 

\subsection{Model:}
In this framework, each goal $g$ corresponds to a reward function $r_g : S \times A \rightarrow R$.
In addition, the agent receives as input the goal at each time-step.
HER parameterizes the action-value function $Q$ with a goal $g$
i.e. $Q^{\pi}(s_t, a_t, g) = \mathbb{E}_{\pi, p(s_0)}[G_t \mid s_t, a_t, g]$.

\subsection{Training:}
HER can be trained using any off-policy reinforcement learning algorithm such as 
DQN, DDPG, NAF, and SDQN. Training proceeds as follows for each episode:
\begin{itemize}
\item Sample a goal $g$, initial state $s_0$, and roll out policy
\item Store trajectory transitions $(s_t \mid g, a_t, r_t, s_{t+1} \mid g)$ with original goal $g$ and other goals $g'$ in replay buffer
\item Sample a mini batch $B$ from replay buffer
\item Perform one step of optimization using algorithm $A$ and mini batch $B$
\en{itemize}

\subsection{Experiments:}
This paper evaluated the model on the following three tasks using the Mujoco simulator 
with a 7-DOF Fetch robotics arm:
\begin{enumerate}
\item Pushing
\item Sliding
\item Pick-and-place
\end{enumerate}

\subsection{My Conclusions: (concerns, follow up, etc)}
\begin{itemize}
\item The actions used in the paper are 4-dimensional and specify the desired position of the gripper at the next time step. 
I am unsure if the gripper will deterministically move to this position at the next time step or not. If so, this seems
a bit simplistic. Can the algorithm accommodate higher dimensional action spaces or must you relay on a method to get
to a specific position ? 
\item What does the "future" version (where $g'$ is from the same episode being replayed and is a state observed after it)
of the algorithm really do? Does it refer to the current state?
\item How could this interact with imitation learning? Could this augment imitation learning somehow? 
\item When else is it useful to parameterize the $Q$ function with goals?
\end{itemize}


